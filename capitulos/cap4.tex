%% ------------------------------------------------------------------------- %%
\chapter{CONSIDERAÇÕES GERAIS}
\label{cap:consideracoes_gerais}

Neste trabalho, analisamos a influência da seleção natural em populações nativas americanas de duas ecorregiões principais: Floresta Amazônica e Andes.

Por meio de múltiplos testes de seleção positiva, análise de enriquecimento gênico, anotação dos genes e vias candidatas e metanálises, hipotetizamos que os genes e vias candidatas à seleção na Amazônia tiveram como pressão seletiva o estilo de vida caçador-coletor, sendo a procura de novidades, o período de escassez alimentícia e a resposta a de patógenos presentes neste ambiente os principais fatores seletivos. Neste contexto, alguns dos traços que outrora foram importantes para a subsistência destas populações podem hoje apresentar um efeito adverso dada a mudança do estilo de vida local, como a introdução de alimentos industrializados e o modo de vida sedentário. Por exemplo, as rotas para o sistema de recompensa do cérebro, que poderiam favorecer maior disposição à procura de novidades (exposição ao risco), podem hoje estar relacionados com os elevados índices de alcoolismo nestas populações. De forma similar, genes que possibilitaram a subsistência em períodos de escassez alimentícia dado à sazonalidade das florestas tropicais, podem estar relacionados com o aumento da obesidade e doenças relacionadas à síndrome metabólica. Em relação ao sistema imunológico, detectamos genes relacionados à seleção de linfócitos T, e genes relacionados à resposta sorológica ao agente transmissor da doença de Chagas, \textit{Trypanosoma cruzi} (\emph{e.g.} \textsl{PPP3CA} e \textsl{DYNC1I1}). Dada à pequena incidência da doença de Chagas nas populações amazônicas, quando comparado à populações andinas, é possível que os nativos amazônicos tenham desenvolvido um mecanismo de defesa ao protozoário \textit{Trypanosoma cruzi}. Torna-se interessante, portanto, a realização de estudos funcionais com os genes aqui identificados para avaliar a taxa de infecção do protozoário.

Nas populações andinas, identificamos cinco genes candidatos principais ─ \textsl{DUOX2}, \textsl{SP100}, \textsl{CLC}, \textsl{TMEM38B}, \textsl{AS3MT}  ─ pelo método PBS, corroborados por sua vez pelo método XP-EHH. Nossos resultados apontaram um mecanismo complexo de adaptação nos nativos andinos, que envolveu os genes \textsl{CLC}, \textsl{SP100} e \textsl{DUOX2}, não encontrados em estudos prévios de adaptação ao ambiente andino. Considerando a diversidade de rotas evidenciadas relacionadas a adaptação exclusiva ao altiplano andino, inferimos que os nativos andinos se adaptaram a esse ambiente de maneiras diferentes e usando estratégias moleculares distintas das de outros povos que vivem em grandes altitudes.

Em conclusão, nosso trabalho identificou  novos genes candidatos tanto na região da floresta amazônica quanto nas terras altas andinas, que muito provavelmente tiveram um papel essencial para adaptação local das populações nativas americanas na Amazônia e nos Andes. Os resultados desta tese abrem portas para outras pesquisas importantes, incluindo estudos funcionais, e levantam uma preocupação no que se refere à saúde pública indígena.
