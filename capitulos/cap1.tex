%% ------------------------------------------------------------------------- %%
\chapter{INTRODUÇÃO GERAL}
\label{cap:introducao}

%% ------------------------------------------------------------------------- %%
\section{TEORIA EVOLUTIVA}
\label{sec:teoria_evolutiva}

\subsection{Evolução por meio da Seleção Natural}

A teoria evolutiva por meio da seleção natural foi proposta inicialmente por Charles Darwin e Alfred Russel Wallace em 1858 \cite{darwin_tendency_1858} em uma conferência realizada em Londres, um ano antes da publicação da obra seminal “A Origem das Espécies por meio da Seleção Natural” de Charles Darwin \cite{darwin_origin_1859}. Em seu próprio livro, Darwin reconheceu que autores anteriores já haviam feito menção à mesma, embora ele tenha o mérito de reunir um vasto conjunto de evidências que suportam especificamente a teoria evolutiva por meio da seleção natural. Assim sendo, Darwin e Wallace propuseram primeiro que as espécies se modificam ao longo do tempo (evolução), onde os descendentes diferem de seus ancestrais ─ descendência com modificação ─, e tais pequenas mudanças, ao longo de um amplo tempo, podem culminar em grandes mudanças. E, segundo, que esta evolução ocorre por meio da seleção natural, em que indivíduos com características que contribuem para a sobrevivência e reprodução, portanto adaptados ao ambiente, tendem a perpetuar, enquanto os demais tendem a ser eliminados \cite{reece_campbell_2013}. 

A priori, tanto Darwin quanto Wallace não abordaram diretamente a evolução humana, sendo somente em 1871 que Darwin discutiu publicamente sobre a evolução humana em seu livro “A Descendência do Homem” \cite{darwin_descent_1871}. Na época, no entanto, Darwin não compreendia, assumidamente, o mecanismo de herança biológica, e tudo indica que não teve conhecimento dos trabalhos de Gregor Mendel, pesquisador contemporâneo de Darwin (embora o contrário não fosse verdade, pois Mendel conhecia o trabalho de Darwin) \cite{fairbanks_mendel_2020}. Mendel publicou seu trabalho que hoje compreende os fundamentos das leis de herança em 1866, ainda que tenha sido reconhecido somente 34 anos depois, após sua morte. A partir da intersecção da evolução e a genética, numerosos avanços e descobertas foram realizadas principalmente ao longo do século XX, sobretudo com contribuições de grandes nomes como Ronald Fisher, Sewall Wright e John Haldane, culminando na base da genética de populações e genética evolutiva, cerne deste trabalho. A seleção natural, no entanto, não é o único mecanismo da evolução, e segundo a teoria neutra \cite{kimura_rate_1969}, tampouco o principal. Portanto, compreender e distinguir os mecanismos que promovem a evolução, além da seleção, é uma etapa crucial nos estudos evolutivos.

\subsection{Mecanismos Evolutivos}

A discussão da evolução dentro do contexto de uma mesma espécie é chamada de microevolução, em contraste com o estudo da evolução entre espécies, denominada macroevolução \cite{jobling_human_2014}. Nesta tese, discute-se apenas a microevolução da nossa espécie (\textit{Homo sapiens sapiens}) e, portanto, deste ponto em diante, sempre que for mencionado evolução (mecanismos e descrições), lê-se microevolução. 

Trazendo para os dias de hoje, do ponto de vista genético, compreende-se por evolução o processo de mudança na frequência alélica ao longo das gerações \cite{vitti_detecting_2013}. As vias pelas quais este processo evolutivo ocorre são conhecidas como mecanismos evolutivos, que podem ser separados em quatro principais: 1) deriva genética, 2) migração, 3) mutação e, por fim,  4) a seleção natural \cite{hancock_detecting_2008}. A ocorrência de qualquer um destes eventos quebra as premissas de um modelo onde não há evolução, denominado Equilíbrio de Hardy-Weinberg \cite{templeton_population_2006}.

A mutação consiste na fonte primária de evolução, ou seja, é ela que introduz alterações diretas no DNA, matéria-prima para a evolução. Uma vez que ela introduz variação no DNA, ela tende a contribuir para a diversidade genética da população \cite{losos_princeton_2014}. A deriva genética pode ser considerada simplesmente como o acaso do processo evolutivo. Em uma população finita, alguns alelos podem mudar de frequência na população de forma aleatória por uma questão puramente amostral \cite{templeton_population_2006}. Em contraste com a mutação, que tende a aumentar a diversidade genética, a deriva tende a diminuir a diversidade genética, e é mais forte em populações menores. A migração consiste no movimento de indivíduos ou gametas de uma população para uma área ocupada por outra população,  que resulta em um fluxo gênico, ou seja, na troca genética entre as populações \cite{templeton_population_2006}. Na genética de populações, ambos os termos migração e fluxo gênico são utilizados de forma equivalente.

Por fim, há a seleção natural, que pode atuar de forma direcional (positiva/negativa) ou balanceadora. A seleção positiva consiste no aumento da frequência de um ou mais alelos, que por sua vez promovem o aumento da sobrevivência, da fertilidade, ou ambos. A seleção negativa ocorre quando um alelo se torna prejudicial ao organismo, diminuindo a sobrevivência ou fertilidade, e sua frequência diminui e tende a desaparecer rapidamente dentro de uma população. Sugere-se que a seleção negativa seja a mais comum dos tipos de seleção que ocorre nas populações \cite{pouyet_background_2018, salas_natural_2019}. A seleção balanceadora ocorre quando um genótipo heterozigoto é vantajoso, onde o equilíbrio entre diferentes alelos em um gene propicia um fenótipo adaptativo.

Atualmente, dispõe-se de mais métodos estatísticos para detecção da seleção positiva, em contraste com a seleção negativa ou balanceadora. Parece haver também um maior interesse (ou em consequência disso) na seleção positiva, uma vez que alelos que tenham sido selecionados possuem uma conexão forte com regiões funcionais do genoma que também são de interesse da área médica. 

\subsection{Métodos de Detecção da Seleção Natural}

Conforme discutido acima, tanto os alelos produzidos por novas mutações, quanto alelos pré-existentes, podem estar sujeitos à seleção natural. Quando isso acontece, ambos deixam marcas no DNA que nos permite inferir quais alelos foram alvo de seleção em um passado recente ou distante.

Dentre as diversas técnicas para inferência de seleção, algumas das abordagens mais comuns incluem os métodos baseados em sítios segregantes (\emph{e.g.} D de Tajima), no desequilíbrio de ligação (\emph{e.g.} iHS, XP-EHH), na distância genética (\emph{e.g.} Fst, PBS), bem como métodos compostos \cite{vitti_detecting_2013}. Tais métodos são eficazes, quando utilizados em conjunto e também com simulações demográficas ou permutações, para detecção de seleção positiva principalmente monogênica \cite{nielsen_recent_2007}. Para detecção de adaptação poligênica, no entanto, outras abordagens são utilizadas, como, por exemplo, análises de enriquecimento gênico e de super-representação \cite{wang_web-based_2013,watanabe_functional_2017,liao_webgestalt_2019}.

\subsubsection{Métodos baseados no desequilíbrio de ligação}

Os métodos baseados no desequilíbrio de ligação, usualmente denominados como \textit{selective sweeps} (em português, varreduras seletivas), possuem como princípio o efeito \textit{genetic hitchhiking} (ou “efeito carona”), proposto inicialmente em 1974 \cite{smith_hitch-hiking_1974}. De forma resumida, compreende-se por \textit{genetic hitchhiking} quando um alelo sob forte pressão seletiva, bem como seus alelos neutros “vizinhos”, aumentam de frequência na população em decorrência da seleção. Como resultado, há uma perda da diversidade genética (heterozigosidade) no haplótipo, gerando uma extensão de homozigose do haplótipo. O primeiro método baseado no desequilíbrio de ligação ─ \textit{Extended Homozygosity Haplotype} (EHH) \textit{Statistics} ─ explora essa extensão de homozigosidade para inferir seleção \cite{sabeti_detecting_2002}. Este tipo de abordagem é extremamente importante na genética evolutiva, visto que através das variantes neutras ligadas aos alelos sob seleção, torna-se possível detectar estes últimos, mesmo sem saber previamente sua localização genômica (nesta era da genômica, este evento também é referido como varredura adaptativa) \cite{novembre_human_2012,stephan_selective_2019}.

A partir da estatística EHH, outros métodos derivados dela foram implementados ao longo dos anos que a sucederam, como Integrated Homozygosity Statistics (iHS) \cite{voight_map_2006}, cross-population EHH (XP-EHH) \cite{sabeti_genome-wide_2007}, e iHS across populations (Rsb) \cite{tang_new_2007}. 

Cada um dos métodos mencionados acima possui suas vantagens e desvantagens, e todos utilizam diretamente ou indiretamente uma variação da estatística EHH, diferindo na abordagem da respectiva aplicação. O método iHS, por exemplo, identifica regiões de extensão de homozigose, diferindo os alelos ancestrais e derivados, em uma mesma população (intra-populacional), considerado robusto para \textit{sweeps} incompletos \cite{vitti_detecting_2013}. Ambos os métodos XP-EHH e Rsb são inter-populacionais, e por meio da comparação com a EEH de uma população próxima, tem maior poder para identificar \textit{sweeps} próximos ou já fixados \cite{suzuki_statistical_2010}. Normalmente, iHS e XP-EHH são utilizados em conjunto nas análises de seleção.

\subsubsection{Métodos baseados na diferenciação populacional}

A métrica mais simples e difundida para mensurar a diferenciação populacional entre duas populações consiste no índice de fixação (Fst). Esta também é uma das métricas mais utilizadas na genética de populações, sendo desenvolvida independentemente por Sewall Wright e Gustave Malécot no final da década de 40 \cite{malecot_les_1948,wright_genetical_1949}. O Fst, como originalmente proposto, mensura a distribuição de variação genética entre subpopulações, comparando a diversidade genética dentro das subpopulações à diversidade genética da população total. Pode-se, contudo, adaptar a fórmula para comparação entre duas populações (Fst par-a-par, ou \textit{pairwise Fst}) para utilizá-la como medida de distância genética \cite{bhatia_estimating_2013,jobling_human_2014}.

O princípio dos métodos baseados na distância genética parte do pressuposto de que, quando a seleção está atuando em um locus dentro uma população, mas não dentro de outra população próxima, a frequência alélica entre estas populações deve variar e mostrar-se maior na população sob pressão seletiva \cite{vitti_detecting_2013}. Partindo deste princípio, surgiu o primeiro teste com o objetivo de detectar sinais de seleção natural utilizando o Fst ─ o Lewontin-Krakauer test (LKT) ─, que foi publicado em 1973 \cite{lewontin_distribution_1973}. A partir de então, principalmente com o avanço da tecnologia e aumento dos dados disponíveis, diversos outros testes foram e continuam surgindo \cite{rees_genomics_2020}.

Dentre os métodos baseados na distância genética, destaca-se aqui os métodos Locus-Specific Branch Length metric (LSBL) \cite{shriver_genomic_2004} e Population Branch Statistics (PBS) \cite{yi_sequencing_2010}. Neles, calcula-se o Fst par-a-par entre pelo menos três populações: a população-alvo e uma população próxima (ditas populações-irmãs), bem como uma população externa (outgroup), permitindo deste modo isolar o ramo de diferenciação na população-alvo.

\subsubsection{Adaptação poligênica}

Apesar dos testes para detecção de seleção positiva discutidos acima serem muito eficientes para atingir seus objetivos, cabe ressaltar que os mesmos possuem como foco sítios monogênicos, uma vez que identificam loci independentes entre si e com fortes sinais de seleção \cite{pritchard_adaptation_2010}. Contudo, compreende-se que a maioria dos fenótipos biológicos são poligênicos, ou seja, vários loci atuam em conjunto a fim de contribuir para um dado fenótipo. Nesse contexto, quando considerados de forma individual, apresentam baixa frequência alélica em cada loci \cite{pritchard_genetics_2010,visscher_10_2017}. Exemplos de fenótipos poligênicos incluem altura, peso, pigmentação e fertilidade.

Alguns dos métodos para detecção da interação entre genes, que possivelmente atuam em conjunto em um mesmo fenótipo (\emph{i.e.} epistasia), atuam através da verificação de interação estatística entre variantes genéticas intra- ou inter-populacional, quer seja por métodos convencionais de correlação que utilizam regressão linear ou logística, respectivamente \cite{wan_boost_2010,ueki_improved_2012}, ou mesmo estatísticas de correlação desenvolvidas especificamente para estudos genéticos \cite{climer_custom_2014,climer_allele-specific_2014}. Contudo, adaptação poligênica não pode nem deve ser deduzida apenas por meio da identificação de sinais de epistasia. Diferentemente da detecção de sinais de seleção positiva monogênica, ainda não existem métodos convencionais estabelecidos para detecção de seleção poligênica \cite{pritchard_genetics_2010} e, ainda, boa parte da aplicação dos métodos desenvolvidos até a presente data têm sido alvos de críticas \cite{sohail_polygenic_2019,refoyo-martinez_how_2020}. 
Métodos para detecção de adaptação poligênica frequentemente têm utilizado bancos de dados funcionais e fenotípicos, como GWAS Catalog \cite{buniello_nhgri-ebi_2019}, GO \cite{ashburner_gene_2000,thegeneontologyconsortium_gene_2019}, KEGG \cite{kanehisa_kegg_2000,kanehisa_new_2019}, entre outros. Outro método baseado em redes (subgrupos), que também adota informações das categorias biológicas presentes em banco de dados públicos, foi desenvolvido a fim de verificar de forma mais sensível sinais de adaptação poligênica \cite{gouy_detecting_2017}. 

Em contraste com outros métodos que requerem dados fenotípicos ou ambientais associados à amostra (\emph{e.g.} \citeauthor{hancock_human_2010}, \citeyear{hancock_human_2010}; \citeauthor{gunther_robust_2013}, \citeyear{gunther_robust_2013}), ou mesmo demográficos (\emph{e.g.} \citeauthor{racimo_detecting_2018}, \citeyear{racimo_detecting_2018}), os métodos que utilizam categorias biológicas, como a análise de enriquecimento gênico (GSEA, do inglês, \textit{gene set enrichment analysis}) \cite{subramanian_gene_2005} ou como a análise de super representação (ORA, do inglês, \textit{over representation analysis}) \cite{khatri_ten_2012}. Por meio dessa abordagem,  atribui-se um valor a cada gene (\emph{e.g.} Fst, ou índices de teste de seleção), e testa-se para verificar se um conjunto de genes com os valores mais altos estão super representados em vias ou fenótipos de banco de dados, considerando o pool gênico da amostra \cite{barghi_polygenic_2020}.

Dada a reconhecida escassez de estudos de adaptação poligênica em nativos americanos \cite{mendes_history_2020}, e que tais métodos têm sido frequentemente aplicados em artigos recentes de adaptação local \cite{bergey_polygenic_2018,harrison_natural_2019,hsieh_exome_2017,lopez_genomic_2019,reynolds_comparing_2019}, optamos por também aplicar métodos de adaptação poligênica no presente estudo.

\section{ADAPTAÇÃO LOCAL NA AMÉRICA DO SUL}

O povoamento da América é um tema amplamente estudado e, desde o início, tem sido alvo de intenso debate na comunidade científica. As diversas áreas da ciência como arqueologia, paleontologia, paleoantropologia, paleoclimatologia, bem como a genética evolutiva, frequentemente se contrapõem. Logo, há um esforço recorrente da comunidade em conciliar os achados dessas diferentes áreas. 

Ao passo que a ciência evolui, como, por exemplo, com o advento de novas metodologias, aquisição e disponibilização de dados, principalmente dados de DNA antigo (aDNA, do inglês, \textit{ancient DNA}), nossa compreensão sobre o povoamento torna-se um pouco mais clara, ainda que provavelmente estejamos longe de um consenso geral. Para discorrer sobre este tema, portanto, é importante salientar que este é um campo com mudanças recorrentes, bem como concepções diferentes entre os próprios pesquisadores da área. Sabendo disso, no texto a seguir adotamos como consenso o delineamento realizado por Sutter em um extensivo artigo de revisão \cite{sutter_pre-columbian_2020}, apresentado aqui de forma a atingir dois objetivos principais: i) introduzir de forma simples e direta os principais pontos para compreensão do povoamento da América, com enfoque na América do Sul, e  ii) discorrer sobre suas consequências para os estudos de adaptação local. 

A hipótese mais plausível para o povoamento das Américas atribui sua entrada pela província da Beríngia, uma região que conectava a Sibéria ao Alasca e noroeste do Canadá, hoje submersa (Estreito de Bering), mas que ficou exposta durante o último máximo glacial (entre 17 e 24 mil anos atrás), formando uma ponte que não apenas possibilitou a passagem dos seres humanos e outros animais, como provavelmente também serviu de abrigo durante esse período tão hostil \cite{hoffecker_beringia_2016,sutter_pre-columbian_2020}.

Este abrigo se explica pelas características climáticas e vegetativas propícias para o refúgio humano, apontadas por estudos paleoambientais \cite{wooller_new_2018}, e também porque duas grandes geleiras na entrada da América bloqueavam a passagem para o continente. Estima-se que os povos ali presentes permaneceram nesta região por aproximadamente 5 a 8 mil anos AP (antes do presente) \cite{fagundes_mitochondrial_2008}. Neste período, provavelmente acumularam mutações que hoje estão presentes em boa parte dos nativo-americanos, como, por exemplo, novos haplogrupos mitocondriais \cite{tamm_beringian_2007}, ou novas variantes com possível maior valor adaptativo, como identificado no gene \textsl{FADS} \cite{amorim_genetic_2017}. 

Partindo do noroeste asiático e da Sibéria, os primeiros nativos-americanos então se diversificaram de uma população ancestral homogênea, separando-se por volta de 22 a 18 mil anos AP em dois ramos principais: o grupo dos beringianos antigos (AB, do inglês, \textit{Ancient Beringians}) \cite{moreno-mayar_terminal_2018,moreno-mayar_early_2018}, e o grupo dos nativos-americanos ancestrais, que por sua vez se dividiram em nativos-americanos do Sul, também denominados como ANC-A (Ancestral A), e nos nativos-americanos do Norte, ou ANC-B (Ancestral-B) \cite{raghavan_population_2015,moreno-mayar_early_2018,posth_reconstructing_2018,scheib_ancient_2018}.

Os primeiros estudos com múltiplas amostras de aDNA da América do Sul com alta cobertura de sequenciamento surgiram em 2018 \cite{posth_reconstructing_2018,moreno-mayar_early_2018}. Apesar de algumas diferenças, ambos concordam que o povoamento da América do Sul ocorreu de forma extremamente rápida, como já anteriormente proposto \cite{llamas_ancient_2016}. As rotas mais prováveis para migração adentro da América do Sul percorridas pelos nativos-americanos foram pela costa do Pacífico Norte (NPC, do inglês, \textit{North Pacific coast}) e o corredor livre de gelo (IFC, do inglês, \textit{ice-free corridor}), formada entre as duas grandes geleiras \cite{potter_current_2018}, sendo que esta última tornou-se viável como rota apenas por volta de 13 mil anos AP \cite{pedersen_postglacial_2016}.

Apesar dos avanços no estudo do povoamento da América, ainda há muito a ser estudado para maior compreensão deste tema, principalmente na América do Sul. \citeonline{posth_reconstructing_2018} reconheceram na conclusão do próprio artigo, por exemplo, que uma das limitações do trabalho foi a ausência de dados (aDNA) proveniente de populações amazônicas. Assim como as terras altas andinas, a Amazônia faz parte das principais ecorregiões americanas \cite{antonelli_amazonia_2018}, e abrigam ainda hoje povos nativos-americanos, que, em conjunto com os dados de aDNA, contribuem tanto para o estudo do povoamento e demografia dos nativos-americanos, quanto para os estudos de adaptação local a ambientes que poderiam ser considerados como inóspitos. 

Desde a saída da África até a entrada na América, as populações passaram por numerosos eventos gargalos de garrafa seguidos de efeitos fundadores, haja visto o evidente decréscimo da diversidade genética ao longo deste percurso \cite{prugnolle_geography_2005}. Estima-se que a população ancestral dos nativos-americanos passou por um gargalo de garrafa profundo \cite{fagundes_mitochondrial_2008}, e o mesmo provavelmente ocorreu em todas as divisões que se sucederam dentro da América. Portanto, torna-se mais do que fundamental analisar de forma cautelosa as análises de seleção natural em populações nativo-americanas, não só para descartar os vieses potenciais da deriva genética, como também o fluxo gênico entre populações de ecorregiões diferentes.

Desta forma, tivemos como objetivo investigar sinais de seleção natural em populações nativas de duas das principais ecorregiões Americanas: Amazônia e Andes, visto que ambas as regiões consistem em “laboratórios naturais” para estudo de adaptação local. Para isso, discorremos sobre as características de cada uma destas regiões, e aplicamos os métodos de detecção de seleção natural aqui mencionados, em conjunto outras abordagens mais específicas (e.g. metanálise, simulação demográfica).
