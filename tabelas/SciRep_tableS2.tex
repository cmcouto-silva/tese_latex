\begin{table}[!htbp]
\centering

\begin{tabularx}{\linewidth}{Xrrr}

\toprule
\textbf{Population (n)} & \makecell[r]{\textbf{DUOX2 G allele}\\\textbf{(rs269866)}} & \makecell[r]{\textbf{SP100 C allele}\\\textbf{(rs13411586)}} & \makecell[r]{\textbf{CLC G allele}\\\textbf{(rs440191)}} \\
\midrule

\multicolumn{4}{l}{\textbf{Mesoamerican Lowland (< 2,500 m)}} \\
\addlinespace[0.1em]
Kaqchikel (13) & 0 & 0.042 & 0 \\
Maya (49) & 0 & 0 & 0.138 \\
Mixe (17) & 0.059 & 0.029 & 0.147 \\
Mixtec (5) 0.100 0.100 0.100 \\
Purepecha (1) 0 0 0 \\
Tepehuano (25) & 0.240 & 0.020 & 0.080 \\
Zapotec (43) & 0.068 & 0.114 & 0.182 \\
\textbf{Total (153)} & \textbf{0.068*} & \textbf{0.045*} & \textbf{0.128*} \\
\addlinespace[1em]
\multicolumn{4}{l}{\textbf{South American (Andean) Highland ($\geq$ 4,000 m)}} \\
\addlinespace[0.1em]
Aymara (23) & 0.457 & 0.413 & 0.500 \\
Quechua (40) & 0.400 & 0.388 & 0.425 \\
\textbf{Total (63)} & \textbf{0.420*} & \textbf{0.397*} & \textbf{0.452*} \\
\addlinespace[1em]
\multicolumn{4}{l}{\textbf{South American (Amazonian) Lowland (< 2,500 m)}} \\
\addlinespace[0.1em]
Guahibo (6) & 0 & 0 & 0.250 \\
Guarani (6) & 0.083 & 0 & 0.167 \\
Jamamadi (1) & 0 & 0.500 & 1 \\
Kaingang (2) & 0.500 & 0 & 0 \\
Karitiana (13) & 0 & 0.115 & 0 \\
Kogi (4) & 0 & 0 & 0.375 \\
Maleku (3) & 0 & 0 & 0 \\
Palikur (3) & 0 & 0.167 & 0.500 \\
Parakana (1) & 0 & 0.500 & 0 \\
Piapoco (7) & 0 & 0 & 0.286 \\
Surui (24) & 0 & 0 & 0 \\
Teribe (3) & 0 & 0.333 & 0 \\
Ticuna (6) & 0.167 & 0 & 0.167 \\
Toba (4) & 0.125 & 0.125 & 0.250 \\
Waunana (3) & 0 & 0.167 & 0.500 \\
Wayuu (11) & 0.056 & 0 & 0.056 \\
Wichi (5) & 0.200 & 0 & 0.300 \\
Yaghan (4) & 0.125 & 0.125 & 0.250 \\
\textbf{Total (106)} & \textbf{0.048*} & \textbf{0.053*} & \textbf{0.142*} \\
\bottomrule
\addlinespace[0.3em]

\multicolumn{4}{l}{\textsuperscript{\textbf{*}} Weighted average}\\
\end{tabularx}

\caption{Allelic frequencies by Native American population analyzed in the present study.}
\label{tab:SciRep_tableS2}

\end{table}